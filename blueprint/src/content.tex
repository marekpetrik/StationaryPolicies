% In this file you should put the actual content of the blueprint.
% It will be used both by the web and the print version.
% It should *not* include the \begin{document}
%
% If you want to split the blueprint content into several files then
% the current file can be a simple sequence of \input. Otherwise It
% can start with a \section or \chapter for instance.

\section{Probability Spaces and Expectation}

In this section, we define the basic probability concepts on finite sets.
% Eventually, this section should be replaced by the Mathlib results using probability theory and measure theory.

\subsection{Definitions}

\begin{definition} \label{def:probability-measure}
A \emph{finite probability measure} $p\colon \Omega \to \Real_+$ on a finite set $\Omega$ is any function that satisfies
\[
\sum_{\omega \in \Omega } p(\omega) = 1. 
\]
\lean{Findist} \leanok
\end{definition}

\begin{definition}
The set of \emph{finite probability measures} $\Delta(\Omega)$ for a finite $\Omega$ is defined as
\[
\Delta(\Omega) := \left\{ p\colon \Omega \to \Real_+ \mid  \sum_{\omega \in \Omega } p(\omega) = 1 \right\}.
\]
\lean{Delta} \leanok
\end{definition}

\begin{definition} \label{def:probability-space}
A \emph{finite probability space} is $P = (\Omega, p)$, where $\Omega$ is a finite set, $p\in \Delta(\Omega)$, and the $\sigma$-algebra is $2^{\Omega}$.
\lean{Finprob} \leanok
\end{definition}

\begin{definition}
A \emph{random variable} defined on a finite probability space $P$ is a mapping $\tilde{x}\colon \Omega \to \Real$.
\lean{Finrv} \leanok
\end{definition}

\begin{definition}
A \emph{boolean} set is $\mathcal{B} = \left\{ \false, \true \right\}$.
\lean{Bool} \leanok
\end{definition}

\begin{definition}
The \emph{expectation} of a random variable $\tilde{x} \colon \Omega \to \Real$ defined on a probability space $P = (\Omega, p)$ is 
\[
\E \left[ \tilde{x} \right] := \sum_{\omega \in \Omega } p(\omega ) \cdot \tilde{x}(\omega).
\]
\lean{Finprob.expect} \leanok
\end{definition}

\begin{definition}
An \emph{indicator} function $\I \colon \mathcal{B} \to \left\{ 0, 1 \right\}$ is defined for $b\in \mathcal{B}$ as
\[
\I(b) :=
\begin{cases}
1 &\text{if } b = \operatorname{true}, \\
0 &\text{if } b = \operatorname{false}.
\end{cases}
\]
\lean{Finprob.indicator} \leanok
\end{definition}

\begin{definition}
The \emph{probability} of $\tilde{b}\colon \Omega \to \mathcal{B}$ for a probability space $P = (\Omega, p)$ is defined as
\[
\P \left[ \tilde{b} \right] := \E\left[\I(\tilde{b})\right].
\]
\lean{Finprob.probability}
\end{definition}

\begin{definition}
The \emph{conditional expectation} of a random variable $\tilde{x} \colon \Omega \to \Real$ conditioned on $\tilde{b} \colon \Omega \to \mathcal{B}$ on a probability space $P = (\Omega, p)$ is defined as
\[
\E \left[ \tilde{x} \mid  \tilde{b} \right] :=
\frac{1}{\P[\tilde{b}]} \sum_{\omega \in \Omega } p(\omega ) \cdot \tilde{x}(\omega) \cdot \I(\tilde{b}(\omega)),
\]
where $x / 0 = 0$ for each $x\in \Real$ (see \texttt{div\_zero}).
\lean{Finprob.expect_cnd} \leanok
\end{definition}

\begin{definition}
The \emph{random conditional expectation} of a random variable $\tilde{x} \colon \Omega \to \Real$ conditioned on $\tilde{y} \colon \Omega \to \mathcal{Y}$ for a finite set $\mathcal{Y}$ is the random variable $\E \left[ \tilde{x} \mid  \tilde{y} \right]\colon \Omega \to \Real$  defined as
\[
\E \left[ \tilde{x} \mid  \tilde{y} \right](\omega)
:=
\E \left[ \tilde{x} \mid  \tilde{y} = \tilde{y}(\omega) \right], \quad \forall \omega \in \Omega.
\]
\lean{Finprob.expect_cnd_rv} \leanok
\end{definition}

\subsection{Basic Results}

\subsubsection{Unconscious Statistician }

\begin{theorem}[Conditional Law of Unconscious Statistician] \label{thm:unc_stat_cond}
Let $\tilde{x} \colon \Omega \to \Real$ and $\tilde{y} \colon \Omega \to \mathcal{Y}$ be random variables defined on a probability space $P = (\Omega, p)$ and a finite set $\mathcal{Y}$. Then:
\[
  \E \left[ \E \left[ \tilde{x} \mid  \tilde{y} \right] \right]
  =
  \sum_{y\in \mathcal{Y}} \E \left[ \tilde{x} \mid  \tilde{y} = y \right] \cdot
                       \P \left[ \tilde{y} = y \right].
\]
\lean{Finprob.unconscious_statistician_cnd}                     
\end{theorem}
\begin{proof}
  
\end{proof}

\subsubsection{Total Expectations Results}

\begin{theorem}[Law of Total Expectation] \label{thm:total_expect}
Let $\tilde{x} \colon \Omega \to \Real$ and $\tilde{y} \colon \Omega \to \mathcal{Y}$ be random variables defined on a probability space $P = (\Omega, p)$ and a finite set $\mathcal{Y}$. Then:
\[
\E\left[\E\left[\tilde{x} \mid  \tilde{y}\right]\right] = \E \left[ \tilde{x} \right].
\]
\lean{Finprob.total_expectation}
\end{theorem}
\begin{proof}[Proof 1: Simpler properties]
  
\end{proof}
\begin{proof}[Proof 2: More advanced properties]
\begin{align*}
\E\left[\E\left[\tilde{x} \mid  \tilde{y}\right]\right]
&= \sum_{y\in \mathcal{Y}} \E\left[\tilde{x} \mid  \tilde{y} = y \right] \P \left[ \tilde{y} = y \right]  \\
&= \sum_{y\in \mathcal{Y}} \E\left[\tilde{x} \mid  \tilde{y} = y \right] \P \left[ \tilde{y} = y \right]  \\
\end{align*}
\end{proof}


\section{Markov Decision Process and Histories}

The basic probability space as defined as follows.
\begin{definition}[Markov Decision Process]
  A Markov decision process $M := (\mathcal{S}, \mathcal{A}, p, r)$ consists of a finite set of states $\mathcal{S}$, a finite set of actions $\mathcal{A}$, transition function $p\colon \mathcal{S} \times \mathcal{A} \times \Delta(\mathcal{S})$, and a reward function $r \colon \mathcal{S} \times \mathcal{A} \times \mathcal{S} \to \Real$.
\lean{MDP} \leanok
\end{definition}

\begin{definition}[History]
A history $h$ in $\mathcal{H}$ is a sequence of states and actions defined for an MDP $M = (\mathcal{S}, \mathcal{A}, p, r)$ and each horizon $T \in \Nats$:
\[
h := (s_0, a_0, s_1, a_1, \dots , s_T),
\]
where $s_k \in \mathcal{S}$ and $a_k\in \mathcal{A}$ for $k = 0, \dots , T-1$.
\lean{Hist} \leanok
\end{definition}

\begin{definition}[Histories]
  The set of histories $\mathcal{H}_T \colon  \mathcal{H} \to 2^{\mathcal{H}}$ for an MDP $M = (\mathcal{S}, \mathcal{A}, p, r)$, following a history $\hat{h}$, is defined for each horizon $T \in \Nats$ as
  \[
    \mathcal{H}_T :=
    \begin{cases}
        \{ \hat{h} \} &\text{ if } T = 0 \\      
        \left\{ \langle h, a, s \rangle \mid h \in \mathcal{H}_{T-1}, a\in \mathcal{A}, s\in \mathcal{S} \right\} &\text{ if } T > 0.
    \end{cases}
  \]
  Here, $\langle \cdot , \cdot , \cdot  \rangle$ is an append operator to augments the history with the action and state.
  \lean{Histories}
  \leanok
\end{definition}

\begin{definition}[History-dependent randomized policies]
  For an MDP $M = (\mathcal{S}, \mathcal{A}, p, r)$, the set of history-dependent policies is $\Pi_{\mathrm{HR}} := \mathcal{S} \to \Delta(\mathcal{A})$.
  \lean{PolicyHR}
  \leanok
\end{definition}

\begin{definition}[History distribution]
  The probability distribution (\cref{def:probability-space}) $p^{\mathrm{h}}_T \in \Delta(\mathcal{H}_T(\hat{h}))$ over histories $\mathcal{H}_T(\hat{h})$ for an MDP $M = (\mathcal{S}, \mathcal{A}, p, r)$ and $\pi \in \Pi_{\mathrm{HR}}$ is defined for each $T \in \Nats$ and $h\in \mathcal{H}_T(\hat{h})$ as
  \[
    p^{\mathrm{h}}_T(h) :=
    \begin{cases}
      1 & \text{ if } T = 0 \wedge h = \hat{h}, \\
      0 & \text{ if } T = 0 \wedge h \neq  \hat{h}, \\
      p_{T-1}(h') \cdot \pi(h',a) \cdot  p(s, a , s') &\text{if } T > 1 \wedge h = \langle h', a, s' \rangle \wedge h' = \langle \cdot , \cdot , s \rangle.
    \end{cases}
  \]
  \lean{HistDist}
  \leanok
\end{definition}

\begin{definition}[History reward]
  The reward $r_{\mathrm{h}}\colon \mathcal{H} \to \Real$ for a history $h \in  \mathcal{H}$,
  \[
   h = (s_0, a_0, s_1, \dots , s_t), 
  \]
  is defined as
  \[
   r^{\mathrm{h}}(h) := \sum_{k=0}^{t-1} r(s_k, a_k, s_{k+1}).
 \]
 \lean{reward}
 \leanok
\end{definition}

\begin{definition}[History expectation]
  For each $T\in \Nats$, $\pi\in \Pi_{\mathrm{HR}}$, a history $\hat{h}\in \mathcal{H}$ and a random variable $\tilde{x}\colon \mathcal{H} \to \Real$ is defined as
  \[
   \E^{\pi, \hat{h}, T} [\tilde{x}]  := \sum_{h\in \mathcal{H}_T(\hat{h})} p^{\mathrm{h}}(h) \cdot x(h).
 \]
 \lean{}
\end{definition}

\section{History-Dependent DP}

Here, we derive dynamic programming equations for histories.

\section{Markov Value Functions}


\section{Markov Policies}

\section{Turnpikes}

%%% Local Variables:
%%% coding: utf-8
%%% mode: LaTeX
%%% TeX-master: "print"
%%% TeX-engine: xetex
%%% End:
